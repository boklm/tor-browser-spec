%\documentclass{llncs}
\documentclass[letterpaper,11pt]{llncs}
%\documentclass{article} % llncs

\usepackage{usenix}
\usepackage{url}
\usepackage{amsmath}
\usepackage{epsfig}
\usepackage{epsf}
\usepackage{listings}

%\setlength{\textwidth}{6in}
%\setlength{\textheight}{8.4in}
%\setlength{\topmargin}{.5cm}
%\setlength{\oddsidemargin}{1cm}
%\setlength{\evensidemargin}{1cm}

\begin{document}

\title{The Future of HTTP and Anonymity on the Internet}

\author{Georg Koppen \\ The Tor Project, Inc \\ georg@torproject.org}
\author{Mike Perry \\ The Tor Project, Inc \\ mikeperry@torproject.org}

%\institute{The Internet}

\maketitle
\pagestyle{plain}

\begin{abstract}


\end{abstract}
 

\section{Introduction}

% XXX: Describe our organization? The Tor Project, Inc is a non-profit...

% XXX: In this position paper, we describe the current and potential issues with

Dangers and opportunities with resepect to browsing the Internet anonymously are
often tied to the browser itself and not its underlying transport protocols:
canvas fingerprinting, plugin enumeration and linking users via the DOM storage
are just a few of the means the browser offers for trying to single users out.
And even things like cookies and referrers, although belonging strictly speaking
to HTTP, the transport protocol powering the web, are usually seen in the
context of the browser itself as its additional policies shape the particular
tracking potential of these and other transport related features.
This is not much different for features in HTTP/2 although, compared to
HTTP/1.1, it has a growing list of tracking risks that should be addressed in
the specification itself. We discuss some of them below proposing ways to take
these and other risks into account in future HTTP specifications.

Apart from the dangers we just hinted at, beginning with HTTP/2 opportunities
emerge as well: Using HTTP/2's flow control could make it easier to defend
against adversaries sniffing a user's encrypted traffic and trying to extract
information out of it by means of website traffic fingerprinting. We discuss
current limitations and potential improvements for HTTP/3.

\section{A Short Tracking Guide(1)}


If we are talking about tracking on the Internet then we mainly have third-party
tracking in mind. In this scenario the attacker has basically two mechanisms
available: identifier based tracking (e.g. using cookies or cache cookies) and
fingerprinting a user's device or environment.

Additionally, we may encounter powerful parties that see a lot of a user's
traffic due to being in a privileged position (e.g. search engines). They don't
necessarily need to bother with third party tracking and would still be able to
learn a lot of a user's details by correlating traffic which is endangering her
anonymity.

The defenses we develop in Tor Browser are:

1) Binding identifiers to the URL bar domain. This is retaining functionality
while preventing cross-origin identifier linkability: saving third party
identifiers (e.g. via DOM storage) in a URL bar domain context does not make
them available in a different URL bar domain context.
2) Making users as uniform as possible while not breaking functionality.
3) Providing a "New Identity" button that is clearing all browser state and
giving the user a clean, new session.


\subsection{User Tracking with HTTP}

\subsection{Re-Using Connections}
Coalescing connections might allow tracking users across origins just by means
of HTTP. Together with a long keep-alive this might make it easy to correlate a
lot of cross-domain traffic of a privacy conscious user even if she has
JavaScript and third-party cookies disabled. Granted, having this feature in
HTTP/2 is a big deal especially with respect to CDNs. But still we think
allowing implementers to provide means to mitigate the issue directly in the
specification seems worthwhile to do. This does not imply avoiding coalescing
connections in the first place. Not at all. One could think about proposing a
middle ground safe-guarding privacy while still providing advantages speed- and
resource-wise: connections should not be reused across different URL bar
domains.

\subsection{Timing Side-Channels}

PING and SETTINGS frames are acknowledged immediately by the client which might
give servers the option to collect information about a client via timing
side-channels. It is true, there are other means an attacker could use for the
same purpose but they are either visible in the browser UI or users can disable
them. As a countermeasure the specification could at least allow jitter in some
cases (PING frames come to mind). If that is not an option one could specify
that a client may close the connection to prevent timing side-channel attacks.


\section{A Short Website Traffic Fingerprinting Guide}


\subsection{Defending against Website Traffic Fingerprinting with HTTP}

(1) For a detailed explanation including a theroretical background see:
https://www.torproject.org/projects/torbrowser/design/.

\section{Conclusions}


\bibliographystyle{plain} \bibliography{W3C-DNT}

\clearpage
\appendix

\end{document}
