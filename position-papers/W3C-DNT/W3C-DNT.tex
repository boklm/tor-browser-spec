%\documentclass{llncs}
\documentclass[letterpaper,11pt]{llncs}
%\documentclass{article} % llncs

\usepackage{usenix}
\usepackage{url}
\usepackage{amsmath}
\usepackage{epsfig}
\usepackage{epsf}
\usepackage{listings}

%\setlength{\textwidth}{6in}
%\setlength{\textheight}{8.4in}
%\setlength{\topmargin}{.5cm}
%\setlength{\oddsidemargin}{1cm}
%\setlength{\evensidemargin}{1cm}

\begin{document}

\title{Do Not Beg:\\Moving Beyond DNT through Privacy by Design}

\author{Mike Perry \\ The Tor Project, Inc \\ mikeperry@torproject.org}

%\institute{The Internet}

\maketitle
\pagestyle{plain}

\begin{abstract}

The Do Not Track header (henceforth DNT:1) seeks to provide privacy
protections against third party tracking through user request and regulation.
It is our position that while DNT:1 is potentially useful as a purely
informational tool for browser vendors and service providers, enforcement of
the header suffers from a number of issues including covert circumvention,
enforcement jurisdiction, manipulation, regulatory capture, and abuse. Moreover,
every privacy property that DNT:1 aims to provide through regulatory
enforcement can be better provided through technical changes to browser and
network behavior during private browsing modes. We therefore suggest that the
W3C standards body focus on standardizing these technical measures, rather
than attempting to broker negotiations over regulatory policy and law.

\end{abstract}
 
% XXX: Bonus References
% Playing dumb:
%   http://www.forbes.com/sites/kashmirhill/2012/02/16/how-target-figured-out-a-teen-girl-was-pregnant-before-her-father-did/
%   https://www.nytimes.com/2012/02/19/magazine/shopping-habits.html?pagewanted=9
% https://tools.ietf.org/id/draft-mayer-do-not-track-00.txt
% DNT Adoption:
%   http://papers.ssrn.com/sol3/papers.cfm?abstract_id=2152135
%   Apache vs MS shit
%   

\section{Introduction}

% XXX: Describe our organization? The Tor Project, Inc is a non-profit...

In this position paper, we describe the current and potential issues with
DNT:1 and associated regulation, and also describe our prototype browser
implementation\cite{torbrowser} that aims to provide the same third party tracking resistance
properties as DNT:1, but without relying on costly regulation and auditing.

We also believe that third party privacy can become a competitive feature for
browser vendors, Internet service providers, and privacy preserving overlay
networks.

\section{Overview of DNT:1}

The Do Not Track header seeks to provide users with a uniform mechanism to
opt-out of third party tracking. Third party content elements are supposed to
honor the header by declining to set cookies and record user activity on their
servers. The draft standard\cite{DNT-draft} states that first party sites do
not need to alter behavior with respect to the header. It also states a number
of exceptions where third parties may still choose to retain and analyze data.

\subsection{Benefits of DNT:1}

The primary benefit of the Do Not Track header is that it provides a strong
signal to browser vendors and websites with respect to their users'
interest in privacy. Within a few months of the header's appearance, 7\% of
desktop and 18\% of mobile Firefox users dug through the Firefox privacy
settings to enable it.\cite{DNT-adoption}

However, despite the value of sizing the market segment for frictionless
privacy enhancing web technologies, it is very likely that DNT:1 will become a
total disaster once the transition to regulatory enforcement draws near.

\subsection{Shortcomings and Dangers of DNT:1}

The primary shortcoming of DNT:1 is that it in no way alters the behavior of
numerous browser technologies that enable and facilitate third party tracking,
and instead relies entirely on ad-hoc auditing and potentially even regulatory
enforcement.

Should strict auditing and direct regulatory requirements be enforced in some
jurisdictions, it is very likely that at least some portion of the advertising
industry would relocate to more favorable jurisdictions. In fact, they would
be incentivized to do so, since it would allow them to offer advertising
services at more favorable rates than their competitors who do not.

Similarly, it introduces serious risks of regulatory capture, especially in
jurisdictions where advertising is able to wield considerable political
influence over the selection of elected officials.

To answer these concerns, some DNT:1 advocates claim they favor "Carrot and
Stick" incentive schemes that do not involve direct regulation, but instead
will rely on web crawls to determine suspicious third party
activity\cite{fourthparty}. Their claim is that violators can be added to an
always-on adblocker filter, and good actors could even be given immunity from
data breach notification requirements and related privacy regulations.

However, without changes to the underlying browser technologies, there are
simply too many ways for advertisers to covertly encode identifying
information in third party elements. Even seemingly innocuous changes such as
minor Javascript and CSS alterations across multiple elements can be used to
encode covert identifiers that are stored in the browser cache for use as
third party tracking cookies. This doesn't even begin to scratch the surface
of covert third party identifier storage and supercookie vectors, let alone IP
address and fingerprinting-based vectors, all of which we will discuss in more
detail in later sections.

Further, behavioral targeting can be made very subtle, and difficult to
distinguish from random chance. For example, Target has begun taking great
pains to obscure behavioral targeting in its catalogs, to avoid alienating
customers. Their targeted advertisements are still present, but they are
merely blended with off-target messaging to provide a false sense of security
and privacy\cite{target}. It is extremely likely that such techniques will be
employed by bad actors in the third party advertising world as well.
 
Further still, because of the various exemptions allowed in the DNT:1
standard, it is hard for users to know when the header is being honored, and
if their activity is still being recorded, exchanged, and sold.

\subsection{Hidden Costs of DNT:1}

We believe that DNT:1 has seen such favorable adoption by browser vendors
because of the ease of deployment for them. Adding a single HTTP header is
substantially simpler than devoting research and development resources into
addressing the network adversary in private browsing modes.

However, DNT:1 merely shifts the costs of privacy development and enforcement
off of the browser vendors and onto every other party involved in the Internet
economy, as well as onto new parties who were previously not involved (such as
auditors, vigilantes, and governmental regulators).

Further, DNT:1 demands that standards organizations such as the W3C shift
gears away from producing and reviewing technical standards to instead
broker policy deals between regulators, legislators, and industry.

% XXX: Possibly too adversarial:
We believe that standards bodies and regulatory agencies shouldn't be wasting
resources asking themselves how, when, and why advertisers don't obey DNT:1.
Instead, they should be asking themselves why browser vendors whose revenue
streams are often directly related to advertising markets continue to deploy
technologies that facilitate and encourage covert third party tracking with no
technical alternatives, even when their users enable their so-called "private
browsing modes".

\section{Do Not Track through Privacy By Design}

Remarkably, the very same third party tracking resistance properties suggested
by the DNT:1 draft standard are possible through a combination of browser and
network behaviors.

All of these properties flow from a very simple core idea: two different first
party domains should not be able to link or correlate activity by the same
user, except with that user's explicit consent.

Initially, consent can be interpreted as link-click navigation. However, as
federated login technologies such as web-send\cite{web-send} and
Persona\cite{Persona} (formerly BrowserID) evolve, link-click based tracking
vectors (such as the Referrer header) can be reduced to the point where they
are easily visible to experts.

To understand the scope of the changes to the browser and network service
providers to provide third party tracking resistance, we need to break down
the problem into roughly four main areas of linkability and privacy:
identifier sources, fingerprinting sources, disk activity, and IP address
utilization.

\subsection{Browser Behavior: Identifier Sources}

Obviously, the primary vector through which third party tracking operates is
the third party cookie. 

Mozilla has a wonderful example of a first party isolation improvement written
by Dan Witte and buried on their wiki\cite{thirdparty}. It describes a new
dual-keyed origin for cookies, so that cookies would only be transmitted if
they matched both the top level origin and the third party origin involved in
their creation. Thus, third party features could still function, if the user
authenticated to that third party within the context of their first party url
bar domain (perhaps using Mozilla's Persona, for example).

With respect to cache identifiers, the earliest relevant example of isolation
work is SafeCache\cite{safecache}. SafeCache eliminates the ability for 3rd
party content elements to use the cache to store identifiers across first
party domains. It does this by limiting the scope of the cache to the origin
in the url bar origin. This has the effect that commonly sourced content
elements are fetched and cached repeatedly, but this is the desired property.
Each of these prevalent content elements can be crafted to include unique
identifiers for each user, in order to track users who attempt to avoid
tracking by clearing cookies.

Other identifier storage mechanisms that require such isolation include
HTTP Auth, window.name, DOM Storage, IndexedDB, SPDY, HTTP-Keepalive, and
cross-domain automated redirects. In Tor Browser\cite{torbrowser}, we either
disable or isolate these technologies.

Properly isolating browser identifiers to the first party domain also has
other advantages as well. With a clear distinction between 3rd party and first
party cookies, the privacy settings window could have a user-intuitive way of
representing the user's relationship with different origins, perhaps by using
only the favicon of that top level origin to represent all of the browser
state accumulated by that origin. The user could delete the entire set of
browser state (cookies, cache, storage, cryptographic tokens, and even
history) associated with a site simply by removing its favicon from their
privacy info panel.

\subsection{Browser Behavior: Fingerprinting Sources}

After identifier isolation, the next source for covert tracking is through
browser fingerprinting. Advertising networks can probe various browser
properties known to differ widely in the userbase, thus constructing an
identifier-free mechanism of tracking users.

Unfortunately, just about every browser property and functionality is a
potential fingerprinting target. In order to properly address the network
adversary on a technical level, we need a metric to measure linkability of the
various browser properties that extend beyond any stored origin-related state.

The Panopticlick project by the EFF provides us with this
metric\cite{panopticlick}. The researchers conducted a survey of volunteers
who were asked to visit an experiment page that harvested many of the above
components. They then computed the Shannon Entropy of the resulting
distribution of each of several key attributes to determine how many bits of
identifying information each attribute provided.

While not perfect\footnotemark, this metric allows us to prioritize effort at
components that have the most potential for linkability.

\footnotetext{In particular, we believe it is impossible to eliminate
inter-browser fingerprinting vectors. Instead, fingerprinting metrics and
defenses should focus on distinguishing features amongst a population with the
same user agent. The Panopticlick test is not currently set up to do this.}

This metric also indicates that it is beneficial for us to standardize on
implementations of fingerprinting resistance where possible. More
implementations using the same defenses means more users with similar
fingerprints, which means less entropy in the metric. It is for this reason
(among others) that the Tor Project seeks to share its Firefox-based browser
implementation\cite{torbrowser} with any interested parties.

The fingerprinting defenses deployed by the Tor Browser include reporting the
desktop resolution as the content window size, reporting a fixed set of of
system colors, disabling plugins by default, limiting the number of fonts a
document is allowed to load, and disallowing read access to the HTML5 canvas
without permission.

The DNT:1 header itself is also fingerprinting vector for bad actors if we
allow our users to set it, and the related scandal between Microsoft and
Apache will likely cause us to entirely remove the DNT:1 option from Tor
Browser's privacy preferences as a result.

\subsection{Browser Behavior: Disk Activity}

In addition to protecting against the network adversary, we believe that
private browsing modes should not force the user to go without disk access. The
two defenses are orthogonal, and private browsing mode users should still be
allowed to store history, bookmarks, and even cookies and cache if they so
choose.

Interestingly, a unified toplevel privacy UI could provide easy access to
quickly clear all of these disk records on a per-site basis, using the same UI
window for both tracking privacy and local disk storage.

\subsection{Network Behavior: IP address utilization}

Currently, there are many ways users can obtain a fresh IP address in an
ad-hoc fashion. Users can use open wireless networks or tether to their
phones. In fact, it is common practice for ISPs in many parts of the world to
rotate user IP addresses daily, to discourage servers and to impede the spread
of malware. This is especially true of cellular IP networks.

Obviously, only technically savvy users are likely to take full advantage of
these properties correctly. However, there is no reason why an IP address
allocation approach can't be generalized and standardized. One could imagine
any privacy proxy (perhaps even one provided by your primary ISP) that
intelligently isolates your first party page loads, along with all of their
associated third party content, to a given IP address. By standardizing such a
mechanism, privacy preserving networks can compete on network properties such
as privacy or performance, rather than some combination of network and user
agent. 

The mechanism Tor has chosen to convey this information to the overlay network
is the SOCKS username and password fields. Our plan is for the Tor Browser to
inform the Tor client which network requests correspond to a given first party
URL bar domain. The Tor client will then ensure that all first party loads use
a different path through the Tor overlay network.

In fact, the Tor Project has concluded that it is in the best interests of the
organization to share user agent development and standardization with other
privacy preserving networks, both to reduce our development efforts, and to
lead to a wider browser fingerprint population for our userbase. The German
privacy company JonDos, GmbH has already joined this effort.

\section{Conclusions}

We discussed the Do Not Track header, the privacy properties it seeks to
provide, and its shortcomings. We believe it is possible to provide these very
same privacy properties through privacy by design.

While the DNT:1 header appears to be a simple change on the browser side, it
has numerous hidden costs in terms of regulators, auditors, and server-side
changes, in addition to serious regulatory challenges. We believe that it will
actually be less costly in total to make the equivalent changes to the
browser, and these changes will have the advantage of supporting markets for
privacy proxies and related privacy enhancing technologies.

\bibliographystyle{plain} \bibliography{W3C-DNT}

\clearpage
\appendix

\end{document}
